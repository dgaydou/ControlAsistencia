%%%%%%%%%%%%%%%%%%%%%%%%%%%%%%%%%%%%%%%%%
% Thin Sectioned Essay
% LaTeX Template
% Version 1.0 (3/8/13)
%
% This template has been downloaded from:
% http://www.LaTeXTemplates.com
%
% Original Author:
% Nicolas Diaz (nsdiaz@uc.cl) with extensive modifications by:
% Vel (vel@latextemplates.com)
%
% License:
% CC BY-NC-SA 3.0 (http://creativecommons.org/licenses/by-nc-sa/3.0/)
%
%%%%%%%%%%%%%%%%%%%%%%%%%%%%%%%%%%%%%%%%%

%----------------------------------------------------------------------------------------
%<Plug>Tex_InsertItemOnThisLine PACKAGES AND OTHER DOCUMENT CONFIGURATIONS
%----------------------------------------------------------------------------------------

\documentclass[a4paper, 11pt]{article} % Font size (can be 10pt, 11pt or 12pt) and paper size (remove a4paper for US letter paper)

\usepackage[spanish]{babel}
\usepackage[utf8]{inputenc}
\usepackage[protrusion=true,expansion=true]{microtype} % Better typography
\usepackage{graphicx} % Required for including pictures
\usepackage{wrapfig} % Allows in-line images
\usepackage{amsfonts}
\usepackage{textcomp} % grados celcius y otras yerbas

\usepackage{mathpazo} % Use the Palatino font
\usepackage[T1]{fontenc} % Required for accented characters
\linespread{1.05} % Change line spacing here, Palatino benefits from a slight increase by default

\makeatletter
\renewcommand\@biblabel[1]{\textbf{#1.}} % Change the square brackets for each bibliography item from '[1]' to '1.'
\renewcommand{\@listI}{\itemsep=0pt} % Reduce the space between items in the itemize and enumerate environments and the bibliography

\renewcommand{\maketitle}{ % Customize the title - do not edit title and author name here, see the TITLE block below
\begin{flushright} % Right align
{\LARGE\@title} % Increase the font size of the title

\vspace{50pt} % Some vertical space between the title and author name

{\large\@author} % Author name
\\\@date % Date

\vspace{40pt} % Some vertical space between the author block and abstract
\end{flushright}
}

%----------------------------------------------------------------------------------------
%<Plug>Tex_InsertItemOnThisLine TITLE
%----------------------------------------------------------------------------------------

\title{\textbf{Programa de mejora de la calidad académica}\\ % Title
Registro del horario de dictado de clase y asistencia} % Subtitle

\author{\textsc{USA for Africa} % Author
\\{\textit{Departamento Electrónica}}} % Institution

\date{\today} % Date

%----------------------------------------------------------------------------------------

\begin{document}

\maketitle % Print the title section

%----------------------------------------------------------------------------------------
%<Plug>Tex_InsertItemOnThisLine ABSTRACT AND KEYWORDS
%----------------------------------------------------------------------------------------

%\renewcommand{\abstractname}{Summary} % Uncomment to change the name of the abstract to something else

%\begin{abstract}
%\end{abstract}

%\hspace*{3,6mm}\textit{Keywords:} bujes  % Keywords

%\vspace{30pt} % Some vertical space between the abstract and first section

%----------------------------------------------------------------------------------------
%<Plug>Tex_InsertItemOnThisLine ESSAY BODY
%----------------------------------------------------------------------------------------

\section*{Fundamentación}
El objeto de este proyecto es supervisar los procesos de enseñanza a fin de proseguir con la mejora continua de la calidad académica de la carrera de Ingeniería en Electrónica, abordando los aspectos de asistencia y puntualidad de los docentes, auxiliares y demás personal del departamento relacionado a la gestión académica; como así también la asistencia de los alumnos a clase. 
\subsection*{Antecedentes}
La necesidad de implementar un mecanismo de control de asistencia docente ha sido señala ya hace más de un año en la nota fechada 10 de Junio 2015, suscripta por los consejeros del claustro estudiantil del consejo departamental de Electrónica: Martínez Denis, Acha Lautaro, Pucci Francisco, García Mauricio, Cuenca Antonio, Flores Mauricio y otros estudiantes de la carrera de Ingeniería Electrónica; dirigida al consejo departamental y cuya copia se adjunta al presente proyecto. El escrito destaca que dar cumplimentación al mandato estatutario de supervisar la calidad de los procesos de enseñanza implica como condición prioritaria el cumplimiento del dictado de clases. También se reclama por  la degradación del cursado de determinadas cátedras en virtud de inasistencias docentes. Todo por lo cual propone la aplicación de algún dispositivo biométrio para obtener constancia del acceso de los docentes a sus lugares de trabajo.
El inciso D del acta de reunión del consejo departamental de junio de 2015 dice:\\
\\
''(D) El director informa que hay una nota presentada por los estudiantes que integran este cuerpo avalada por una lista de firmas donde se solicita la adecuación y mejora del método de registro y control de asistencia de docentes y personal que cumple funciones en el departamento. El ing. Grazzini explica que se dio a la petición curso mediante una nota elevada al Secretario académico para que se analicen las posibilidades de implementar el sistema propuesto. Asimismo y relacionado con el tema, el director pone en conocimiento del consejo las novedades respecto al incumplimiento de la circular emitida oportunamente sobre el llenado de los libros de temas.''\\
\\
Consideramos que la implementación del control de asistencia está encuadrado dentro de las atribuciones del gobierno del departamento, por lo que carece de objeto cursar pedidos de autorización o consultar por la implementación del mismo. Está en las atribuciones del consejo departamental velar por el cumplimiento de los deberes de los docentes del departamento y en consecuencia determinar los métodos de control a aplicar.\\
\\
En cuanto a la toma de asistencia será un requisito a partir del año entrante cuando se aplique el regimen de promoción directa.
\subsection*{Finalidad}
Disponer de una base de datos con el registro del horario de inicio y fin del dictado de las asignaturas por parte de los docentes que permita al director de departamento girar mensualmente a la comisión de enseñanza un reporte que dé constancia de las clases efectivamente dictadas y copia del libro de temas correspondiente al periodo junto con la planificación de la cátedra.
Que de esta forma la comisión de enseñanza debe emitir un dictamen en todas las reuniones ordinarias de consejo, dando cuenta de las desviaciones entre la planificación y los temas dictados y su correlación, en caso de corresponder con la falta de dictado de clase; como así también observaciones sobre situaciones de inasistencias colectivas.
\section*{Objetivos Generales}
\begin{itemize}
	\item Lanzar la prueba piloto del sistema a partir del primer cuatrimestre del ciclo lectivo 2017.
\end{itemize}
\section*{Objetivos Particulares}
\begin{itemize}
	\item Adquirir e instalar un dispositivo biométrico integrándolo a los sistemas informáticos gestionados por centro de cómputos, con transferencia de datos automática sin intervención de operador humano.
	\item Reglamentar la operación del sistema
	\item Reglamentar la carga de datos
	\item Reglamentar el proceso de puesta en marcha y periodo de adaptación
	\item Reglamentar el tratamiento de los reportes en las comisiones y en el consejo
\end{itemize}
\section*{Implementación y funcionamiento}
Para la operación del sistema propuesto, se distinguen cuatro aspectos fundamentales
\begin{itemize}
	\item Dispositivo biométrico autónomo de registro de horario enlazado a base de datos del centro de cómputos.
	\item Software para extracción, procesamiento y presentación de reportes.
	\item Actividad regular mensual de la comisión de enseñanza en la evaluación de los reportes y la generación de dictámenes en todas las reuniones ordinarias.
	\item Operación del software bajo la responsabilidad secretario de departamento.
\end{itemize}
\section{Operación del sistema}
En la operación del sistema toman participación diferentes actores del cuerpo de la carrera. El consejo en la supervisión regular de los informes, la comisión de enseñanza en la generación de los reportes, el secretario en la gestión del sistema de software, los docentes registrando con su marca biométrica el inicio y fin del dictado de las asignaturas y los alumnos firmando las planillas de asistencia.
\subsection{Planilla de asistencia}
Es un documento con el listado de los nombres de los alumnos de cada comisión y un espacio para que registren su firma.
En el encabezado habrá un espacio para que el o los docentes que estén a cargo del dictado ese día pongan su nombre y firma. Estas planillas deberán ser puesta a disposición de los alumnos por parte del docente durante los primeros 20 minutos del horario oficial de inicio de la clase o en caso de tardanza superior por la razón que sea, deberán ponerse a disposición de los alumnos inmediatamente para que registren su asistencia firmando. A los 20 minutos el docente retira y guarda la planilla. No se podrán incorporar nuevas firmas una vez retirada la planilla. El docente debe verificar que el número de firmas coincida con los presentes al momento de retirarla. En caso que las firmas excedan a los presentes, tomará lista en forma oral y tachará de la lista los ausentes.\\
\subsection{Registro de inicio y fin del dictado de clase}
Los docentes deberán dejar constancia del horario de inicio y fin del dictado de clase utilizando el dispositivo de registración biométrico dispuesto en el departamento de electrónica.\\
Los docentes tendrán una tolerancia de 15 minutos contados a partir del horario reglamentario de inicio del dictado de clases que le corresponde, a partir de lo cual se computará como inicio tardío del dictado hasta 45 minutos contados de igual forma. Cualquier registro de ingreso posterior se computará como clase no dictada, sin perjuicio de que el docente pueda hacer su descargo por escrito antes de transcurridas 48 horas cuya constancia será agregada en el campo de justificaciones del sistema informático; donde además deberá consignar los temas dictados y la cantidad de alumnos presentes mediante la planilla de asistencia con firma de los mismos.\\
La finalización del dictado de clase deberá quedar registrado en el dispositivo, los registros deberán realizarse a partir de los 5 minutos posteriores a la finalización del horario de dictado de la materia, y hasta 30 minutos. Los registros fuera de esta ventana de tiempo se computarán como como clase no dictada.\\
Todos los registros de un docente que califiquen como clase dictada serán considerados con carácter de declaración jurada como así también las planillas de asistencia entregadas al secretario del departamento.\\
Si transcurridos 25 minutos del horario oficial de inicio de clase, no se hiciera presente ningún alumno de la comisión correspondiente; podrá el docente retirarse, registrar el horario de salida y entregar al secretario del departamento la planilla con su nombre y firma dentro del plazo estipulado a tal fin.\\
\subsection{Gestión administrativa del secretario}
El secretario de departamento será encargado de la gestión operativa del software. Deberá generar los documentos con los resultados de los periodos solicitados por el director del departamento para ser girados a la comisión de enseñanza. Deberá receptar las planillas de asistencias remitidas por los docentes y archivarlas para que estén a disposición de la comisión de enseñanza. Será el único habilitado para editar los campos de comentarios en la base de datos, donde deberá consignar los avisos de inasistencia, licencias y franquicias de los docentes todo en acuerdo con la Ordenanza 474 C.S.U. y sus modificatorias Ord. 671 C.S.U. y Ord. 740 C.S.U.\\
También será el encargado de cargar los registros en el dispositivo biométrico y el software cliente. Deberá actualizar anualmente el calendario del programa con el calendario académico emitido por la secretaría académica e incluír también las fechas no laborables.\\
Deberá también cargar las fechas en que el gremio FAGDUT convoque a paro de actividades.\\
Deberá cargar los suplentes habilitados en cada cátedra.
\subsection{Comisión de enseñanza}
Deberá generar un despacho conteniendo la evaluación del periodo girado por cada cátedra. El despacho será "sin novedad" en aquellos casos en que no se observen discrepancias substanciales entre la planificación y el avance declarado en el libro de temas, y que además no se hayan recibido quejas formales ni avisos de licencias, renuncias o jubilaciones y será informado por el director de departamento al iniciar la sesión.\\
Para cualquier otra situación elaborará un informe acompañado de toda documentación probatoria pertinente y un dictámen o recomendación al respecto; el cual será automáticamente incluído en el temario de la siguiente sesión ordinaria del consejo departamental.
\subsection{Consejo Departamental}
Podrá aprobar o rechazar los dictámenes de la comisión.
\end{document}
