
\section*{Implementación y funcionamiento}
Para la operación del sistema propuesto, se distinguen cuatro aspectos fundamentales
\begin{itemize}
	\item Dispositivo biométrico autónomo de registro de horario enlazado a base de datos del centro de cómputos.
	\item Software para extracción, procesamiento y presentación de reportes.
	\item Actividad regular mensual de la comisión de enseñanza en la evaluación de los reportes y la generación de dictámenes en todas las reuniones ordinarias.
	\item Operación del software bajo la responsabilidad secretario de departamento.
\end{itemize}
\section{Para las reglamentaciones-scratch}
\subsection{de las planillas de asistencia}

Es un documento con el listado de los nombres de los alumnos de cada comisión y un espacio para que registren su firma.
En el encabezado habrá un espacio para que el o los docentes que estén a cargo del dictado ese día pongan su nombre y firma. Estas planillas deberán ser puesta a disposición de los alumnos por parte del docente durante los primeros 20 minutos del horario oficial de inicio de la clase o en caso de tardanza superior por la razón que sea, deberán ponerse a disposición de los alumnos inmediatamente para que registren su asistencia firmando. A los 20 minutos el docente retira y guarda la planilla. No se podrán incorporar nuevas firmas una vez retirada la planilla. El docente debe verificar que el número de firmas coincida con los presentes al momento de retirarla. En caso que las firmas excedan a los presentes, tomará lista en forma oral y tachará de la lista los ausentes. La planilla tendrá una columna impresa con los números de legajo únicamente, los alumnos deberán completar de puño y letra la columna correspondiente a su nombre completo y la columna con su firma.\\
\subsection{de los registros biométricos}
Los docentes tendrán una tolerancia de 15 minutos contados a partir del horario reglamentario de inicio del dictado de clases que le corresponde, a partir de lo cual se computará como inicio tardío del dictado hasta 45 minutos contados de igual forma. Cualquier registro de ingreso posterior se computará como clase no dictada, sin perjuicio de que el docente pueda hacer su descargo por escrito antes de transcurridas 48 horas cuya constancia será agregada en el campo de justificaciones del sistema informático; donde además deberá consignar los temas dictados y la cantidad de alumnos presentes mediante la planilla de asistencia con firma de los mismos.\\
La finalización del dictado de clase deberá quedar registrado en el dispositivo, los registros deberán realizarse a partir de los 5 minutos posteriores a la finalización del horario de dictado de la materia, y hasta 30 minutos. Los registros fuera de esta ventana de tiempo se computarán como como clase no dictada.\\
Todos los registros de un docente que califiquen como clase dictada serán considerados con carácter de declaración jurada como así también las planillas de asistencia entregadas al secretario del departamento.\\
Si transcurridos 25 minutos del horario oficial de inicio de clase, no se hiciera presente ningún alumno de la comisión correspondiente; podrá el docente retirarse, registrar el horario de salida y entregar al secretario del departamento la planilla con su nombre y firma dentro del plazo estipulado a tal fin.\\
\subsection{de la gestión administrativa}
 Deberá generar los documentos con los resultados de los periodos solicitados por el director del departamento para ser girados a la comisión de enseñanza. Deberá receptar las planillas de asistencias remitidas por los docentes y archivarlas para que estén a disposición de la comisión de enseñanza. Será el único habilitado para editar los campos de comentarios en la base de datos, donde deberá consignar los avisos de inasistencia, licencias y franquicias de los docentes todo en acuerdo con la Ordenanza 474 C.S.U. y sus modificatorias Ord. 671 C.S.U. y Ord. 740 C.S.U.\\
También será el encargado de cargar los registros en el dispositivo biométrico y el software cliente. Deberá actualizar anualmente el calendario del programa con el calendario académico emitido por la secretaría académica e incluír también las fechas no laborables.\\
Deberá también cargar las fechas en que el gremio FAGDUT convoque a paro de actividades.\\
Deberá cargar los suplentes habilitados en cada cátedra.
